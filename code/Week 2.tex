\documentclass{beamer}
\usepackage{graphicx}

    \title{Elements of Microeconomics}
    \author{Hank Behaeghel}
    \date{Week 2}

\begin{document}

\maketitle

\begin{frame}{Office Hours Schedule}
    \begin{itemize}
        \item Monday: 9:30-11:00 AM
        \item Wednesday: 10-11:30 AM and 1:30-3:30 PM
        \item Thrusday: 2:00-4:00PM
        \vspace{5mm}
        \item These are subject to change. Please keep an eye for announcements adjusting them.
        \item If NONE of these work, that is ok. Please email me if you'd like to meet and we can set up a time.
    \end{itemize}
\end{frame}
\begin{frame}
    \begin{itemize}
        \item Last week we bagan a short introduction of trade. This weeek we will dive deeper into trade.
        \vspace*{5mm}
        \item Remeber the question: Is trade always benefical?
        \begin{enumerate}
            \item In this class we will not focus on the empirics of this statement as that is for later studies.
            \item By developing the idea \textit{comparative advantage} we will see that trade can make everyone better off.
        \end{enumerate}
    \end{itemize}
\end{frame}

\begin{frame}{Absolute and Comparative Advantage}
    \begin{itemize}
        \item \textit{Absolute Advantage} - the ability to produce a good using fewer inputs than another producer.
        \vspace{5mm}
        \item \textit{Comparative Advantage} - the ability to produce a good at a lower \textbf{opportunity cost} than another producer.
        
        \vspace{5mm}
        \item \textit{Opportunity Cost}:
            \begin{enumerate}
                \item This concept never goes away and relates back to the principle of \textbf{trade-offs}.
                \item Formally, OC is defined as: \textit{whatever must be given up to obtain some item}.
                \item Opportunity cost is more than just forgone earnings,, it can also include things such as time.
            \end{enumerate}
    \end{itemize}
    
\end{frame}

\begin{frame}{Back to the PPF}
    \begin{itemize}
        \item Let us take a look at two individuals who might seek to engage in trade relations.
        \item The following PPFs shows us both Ruby and Frank's output of meat and potatoes in an 8hr work day
    \end{itemize}
    \includegraphics[width = \textwidth]{/Users/hank/Desktop/Fall 2023 Lecturing Materials/Micro/Lecture Screenshots/Ruth and Frank PPF.png}
\end{frame}

\begin{frame}{Determining Trade}
    \begin{itemize}
        \item Steps to Determine Trade
        \begin{enumerate}
            \item Draw the PPFs (you'll have to do this on exams)
            \item Calculate the opportunity costs for producing meat and potatoes respectively for each farmer.
            \item Determine the trade price.
        \end{enumerate}
        \item Using the PPFs and the 8 hour work day we can extract the following information:
            \begin{center}
                \includegraphics[width = 0.8\textwidth]{/Users/hank/Desktop/Fall 2023 Lecturing Materials/Micro/Lecture Screenshots/RFProductions.png}
            \end{center}
    \end{itemize}
\end{frame}

\begin{frame}{Setting the Price of Trade}
    \begin{itemize}
        \item To set the price of trade we must first determine the opportunity costs for both goods for both farmers.
        \vspace{5mm} 
        \begin{itemize}
            \item Knowing the opportunity costs will allow us to determine the \textbf{specialization} of each farmer.
        \end{itemize}
        \vspace{5mm}
        \item The price of trade \textbf{will always line between the 2 opportunity costs} for both parties to gain.
    \end{itemize}
    
\end{frame}

\begin{frame}{Another Example}
    Suppose it requires 10 labor hours for The Riverlands to produce 1 computer and 20 labor hours for The Riverlands to produce 1 sandwich. The Vale can produce 1 computer using 25 labor hours and 1 sandwich using 5 labor hours. Both kingdoms are endowed with 100 labor hours. 
    \vspace{5mm}\begin{enumerate}
        \item Draw the PPF for both kingdoms
        \item Who has the absolute advantage in producing these goods?
        \item Who should produce what and how do you know?
    \end{enumerate}

\end{frame}

\begin{frame}{Strategy for Approaching Any Trade Problem}
    \begin{enumerate}
        \item Figure out the traders and goods.
        \item Draw PPF
            \begin{itemize}
                \item If it is constant OC then PPF is straight line. If not then it is bowed. Be mindful of this.
            \end{itemize}
        \item Determine absolute advantage and comparative advantage.
        \begin{itemize}
            \item This tells you who will trade what.
        \end{itemize}
        \item Determine trade price.
        \begin{itemize}
            \item This is based on the opportunity costs of production for each party. Let the comparative advatnage guide you.
            \item Always remember: \textit{the price of trade will always lie between the two countries opportunity costs}.
        \end{itemize}
    \end{enumerate}
\end{frame}
\end{document}

