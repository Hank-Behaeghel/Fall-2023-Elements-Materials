\documentclass[aspectratio=169]{beamer}
\usetheme{Boadilla}
\usepackage{graphicx}
\usepackage{amsmath}

    \title{Elements of Microeconomics}
    \author{Hank Behaeghel}
    \institute{Johns Hopkins University}
    \date{Week 5}

\begin{document}

\maketitle

\begin{frame}{Review Session and Exam}
    \begin{itemize}
        \item Review session: \textbf{October 3rd, 7:30-9:30 Mudd 26}
        \vspace*{5mm}
        \item Exam 1: \textbf{October 5th, 2023} (i.e. next Thursday)
    \end{itemize}
    
\end{frame}

\begin{frame}{Rubber Meets the Road}
    \begin{itemize}
        \item So far in this course we have primarily discussed theory in a vaccum.There have been some applications such as trade but, our discussions on this have been limited.
        \item Chapter 6 puts the first 5 chapters together and delivers some tangible policy discussions to further our intution.
        \item A third actor will be joining our discussion of markets, \textbf{the government}.
        \item We will examine the topics of producer and consumer surplus as they relate to policy interventions.
    \end{itemize}

\end{frame}

\begin{frame}{Government Policies}
    \begin{itemize}
        \item In this course we will not discuss the nuances of how policies are passed etc., rather simply their outcomes according to the tools you have been given so far.
        \vspace{5mm}
        \item This courses focuses on a few specifc policies:
        \vspace{2mm}
            \begin{enumerate}
                \item \textbf{Price controls}: price floors and price ceilings
                \vspace{1mm}
                \item \textbf{Taxes}: these can be levied on consumers and suppliers (we will discuss these later in the course)
                \vspace{1mm}
                \item \textbf{Subsidies}: these have been discussed in the supply and demand chapter
            \end{enumerate}
    \end{itemize}
    
\end{frame}

\begin{frame}{Price Controls}
    \begin{enumerate}
        \item<1-> \textbf{price ceiling}:\onslide<2-> ~a legal maximum on the price at which a good can be sold.
        \vspace{5mm}
        \item <3->\textbf{price floor}:\onslide<4->~a legal minimum on the price at which a good can be sold.
    \end{enumerate}
    \vspace{5mm}
    \begin{itemize}
        \item<5-> Both of these controls can either be \textit{binding} or \textit{nonbinding}.
        \item<5-> We will explore what this means and its implications.
    \end{itemize}
\end{frame}

\begin{frame}{What Does it Mean to be Binding?}
    \begin{itemize}
        \item If a floor or ceiling is binding it simply means that it restricts the market from reaching equilibrium.
    \end{itemize}
    \begin{center}
    \begin{tabular}[c]{|c|c|c|}
        \hline
        Control & Binding & Not Binding \\
        \hline
        Floor & Above Eq & Below Eq \\
        \hline
        Ceiling & Below Eq & Above Eq \\
        \hline
   \end{tabular}
\end{center}
\end{frame}

\begin{frame}{Price Ceiling Graphically}
    \includegraphics[width = \textwidth]{Rent Control.png}
\end{frame}

\begin{frame}{Price Floor Graphically}
    \includegraphics[width = \textwidth]{Price Floor.png}
\end{frame}

\begin{frame}{Consumer Surplus}
    \begin{itemize}
        \item Every consumer in the market has a price that they are willing to buy at. You and I may have different prices we are willing to pay for good and services for many reasons that are not covered in this course but, that you'll encounter later on.
        \item This max price that someone is willing to pay is often labeled \textit{willingness to pay}.
        \item Consider for a moment the demand curve, as price increases the  $Q_d$ decreases, this illustrates that there are still individuals in the market willing to pay this high price. It does not necessarily that you concede to it.
        \begin{block}{Consumer Surpuls}
            CS is the amount a buyer(s) are willing to pay minus the amount the buyer actually pays for it.
        \end{block}
    \end{itemize}
\end{frame}

\begin{frame}{Producer Surplus}
    \begin{itemize}
    \item Similar to consumers, suppliers have a price that they are willing to supply at. These may differ across supplies.
    \item However, unlike consumers we must consider an extra element to a suppliers decision to bestow goods upon the market, \textbf{cost}.
    \item Here cost should be considered the painters opportunity cost: out-of-pocket \textit{as well as the value they place on their own time}.
    \item This cost is the \textbf{willingness to sell} for a producer. Remember that one is indifferent if the gain from work and the value of outside options are the same.
    \end{itemize}
    \begin{block}{Producer Surplus}
        PS is the amount a seller is paid for a good minus the seller's cost of providing it.
    \end{block}
\end{frame}

\begin{frame}{CS and PS Graphically}
    \begin{center}  
    \includegraphics[scale = 0.5]{PSCS.png}
    \end{center}
\end{frame}
\end{document}